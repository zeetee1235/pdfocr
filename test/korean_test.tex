\documentclass[12pt]{article}
\usepackage{CJKutf8}
\usepackage{geometry}
\usepackage{amsmath}
\geometry{margin=1in}

\title{OCR Test - English and Korean}
\author{PDF OCR Test}
\date{\today}

\begin{document}
\maketitle

\section{Introduction}

This is a test document for OCR with multilingual support.

\begin{CJK}{UTF8}{mj}
이것은 OCR 테스트 문서입니다. 영어와 한글을 동시에 처리할 수 있는지 확인합니다.
\end{CJK}

\section{Lists}

\subsection{English Items}
\begin{itemize}
\item First item in English
\item Second item: numbers 123, 456
\item Third item: symbols test
\end{itemize}

\subsection{Korean Items}

\begin{CJK}{UTF8}{mj}
\begin{itemize}
\item 첫 번째 한글 항목
\item 두 번째 항목: 가나다라
\item 세 번째 항목: 안녕하세요
\end{itemize}
\end{CJK}

\section{Mathematics}

Quadratic formula:
\begin{equation}
x = \frac{-b \pm \sqrt{b^2 - 4ac}}{2a}
\end{equation}

Einstein equation:
\begin{equation}
E = mc^2
\end{equation}

\section{Mixed Text}

\begin{CJK}{UTF8}{mj}
대한민국의 수도는 서울입니다. The capital of Korea is Seoul. 과학기술이 발전하면서 인공지능이 중요해졌습니다.
\end{CJK}

Artificial intelligence has become important.

\section{Conclusion}

\begin{CJK}{UTF8}{mj}
이 문서는 OCR 시스템이 영어와 한글을 동시에 처리할 수 있는지 확인하기 위한 테스트입니다.
\end{CJK}

This document tests whether the OCR system can process both English and Korean.

\end{document}
